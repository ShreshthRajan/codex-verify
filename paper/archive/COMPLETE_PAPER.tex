% ========================================================================
% CodeX-Verify: Multi-Agent Verification of LLM-Generated Code
% Complete LaTeX Document for Overleaf
% ========================================================================

\documentclass[11pt]{article}

% ========================================================================
% PACKAGES
% ========================================================================

% Page layout
\usepackage[margin=1in]{geometry}
\usepackage{setspace}

% Math and algorithms
\usepackage{amsmath, amssymb, amsthm}
\usepackage{algorithm}
\usepackage{algorithmic}
\usepackage{enumerate}

% Tables and figures
\usepackage{booktabs}
\usepackage{multirow}
\usepackage{graphicx}
\usepackage{xcolor}

% References and citations
\usepackage{natbib}
\usepackage{hyperref}
\usepackage{url}

% Formatting
\usepackage{verbatim}
\usepackage{enumitem}

% Custom colors
\definecolor{ForestGreen}{RGB}{34,139,34}

% Theorem environments
\newtheorem{theorem}{Theorem}
\newtheorem{proposition}[theorem]{Proposition}
\newtheorem{lemma}[theorem]{Lemma}
\newtheorem{corollary}[theorem]{Corollary}
\newtheorem{definition}{Definition}

% Custom commands
\newcommand{\checkmark}{\ding{51}}

% Hyperref setup
\hypersetup{
    colorlinks=true,
    linkcolor=blue,
    filecolor=magenta,
    urlcolor=cyan,
    citecolor=blue,
}

% Line spacing
\onehalfspacing

% ========================================================================
% TITLE AND AUTHORS
% ========================================================================

\title{\textbf{CodeX-Verify: Multi-Agent Verification of LLM-Generated Code via Compound Vulnerability Detection and Information-Theoretic Ensemble}}

\author{
    Shreshth Rajan \\
    [Your Institution] \\
    \texttt{your.email@institution.edu}
    % Add co-authors as needed
}

\date{October 2025}

% ========================================================================
% DOCUMENT START
% ========================================================================

\begin{document}

\maketitle

% ========================================================================
% ABSTRACT
% ========================================================================

\begin{abstract}
Large language models generate code with systematic correctness failures: 29.6\% of SWE-bench ``solved'' patches exhibit behavioral incorrectness, 62\% of BaxBench solutions contain vulnerabilities, and traditional static analyzers achieve only 65\% detection accuracy with 35\% false positive rates. Current verification approaches analyze code along a single dimension, missing the complementary nature of bug patterns across correctness, security, performance, and maintainability spaces.

We introduce \textsc{CodeX-Verify}, a multi-agent verification framework with two novel contributions: (1) \emph{information-theoretic foundations} proving multi-agent systems achieve strictly higher mutual information with bug presence than any single agent when detection patterns are non-redundant, and (2) \emph{compound vulnerability detection}, the first formalization of exponential risk amplification for co-occurring vulnerabilities via attack graph theory. Our framework employs four specialized agents that analyze orthogonal bug dimensions (empirical correlation $\rho = 0.05$--$0.25$), with weighted aggregation optimizing the precision-recall frontier.

Rigorous evaluation on 99 samples with perfect ground truth yields 76.1\% true positive rate, matching state-of-the-art Meta Prompt Testing (75\%) while operating purely via static analysis. Comprehensive ablation across 15 configurations proves multi-agent architectures provide 39.7 percentage point improvement over single-agent baselines (32.8\% $\to$ 72.4\%), with the optimal 2-agent pair (Correctness + Performance) achieving 79.3\% accuracy. Real-world validation on 300 Claude Sonnet 4.5-generated patches demonstrates practical deployment at sub-200ms latency.

Our compound vulnerability formalization introduces risk amplification factors $\alpha \in \{1.5, 2.0, 2.5, 3.0\}$ for attack chains, where traditional additive models compute risk as 20 for (SQL injection + credentials) versus our exponential model yielding risk of 300, reflecting 15$\times$ real-world impact. PAC-style sample complexity analysis validates our 99-sample evaluation provides generalization guarantees with 95\% confidence bounds. This work establishes the first rigorous multi-agent framework for code verification, with implications for reducing false positive acceptance in automated software development.
\end{abstract}

\textbf{Keywords:} Multi-agent systems, Code verification, LLM-generated code, Static analysis, Compound vulnerabilities, Information theory, Ensemble learning

\vspace{0.5cm}

% ========================================================================
% NOTE: PASTE CONTENT FROM paper_title_abstract.tex HERE
% ========================================================================

% Starting from Section 1 (Introduction) through Appendix
% Copy from line 56 onwards in paper_title_abstract.tex

% [CONTENT GOES HERE - See instructions below]

% ========================================================================
% BIBLIOGRAPHY
% ========================================================================

\bibliographystyle{plain}
\bibliography{references}

% NOTE: Create references.bib file with BibTeX entries from
% paper_title_abstract.tex (lines 1245-1704 in the \begin{comment} block)

\end{document}

% ========================================================================
% INSTRUCTIONS FOR OVERLEAF:
% ========================================================================
%
% 1. Create new Overleaf project
%
% 2. Create main.tex and paste THIS FILE
%
% 3. In the section marked [CONTENT GOES HERE], paste from
%    paper_title_abstract.tex starting from:
%    "% SECTION 1: INTRODUCTION"
%    through
%    "% END OF APPENDIX"
%
% 4. Create references.bib file and paste BibTeX entries from
%    paper_title_abstract.tex (the content inside \begin{comment}...\end{comment})
%
% 5. Compile: pdflatex -> bibtex -> pdflatex -> pdflatex
%
% Expected output: ~18-19 page paper with:
%   - Title & Abstract
%   - 8 numbered sections (Intro through Conclusion)
%   - References
%   - Appendix (A-J)
%
% ========================================================================
